%%%%%%%%%%%%%%%%%%%%%%%%%%%%%%%%%%%%%%%%%
% "ModernCV" CV and Cover Letter
% LaTeX Template
% Version 1.11 (19/6/14)
%
% This template has been downloaded from:
% http://www.LaTeXTemplates.com
%
% Original author:
% Xavier Danaux (xdanaux@gmail.com)
%
% License:
% CC BY-NC-SA 3.0 (http://creativecommons.org/licenses/by-nc-sa/3.0/)
%
% Important note:
% This template requires the moderncv.cls and .sty files to be in the same 
% directory as this .tex file. These files provide the resume style and themes 
% used for structuring the document.
%
%%%%%%%%%%%%%%%%%%%%%%%%%%%%%%%%%%%%%%%%%

%----------------------------------------------------------------------------------------
%	PACKAGES AND OTHER DOCUMENT CONFIGURATIONS
%----------------------------------------------------------------------------------------
%-- coding: UTF-8 --


\documentclass[12pt,a4paper,sans]{moderncv} % Font sizes: 10, 11, or 12; paper sizes: a4paper, letterpaper, a5paper, legalpaper, executivepaper or landscape; font families: sans or roman

\usepackage[UTF8]{ctex}

\moderncvstyle{classic} % CV theme - options include: 'casual' (default), 'classic', 'oldstyle' and 'banking'
\moderncvcolor{green} % CV color - options include: 'blue' (default), 'orange', 'green', 'red', 'purple', 'grey' and 'black'

\usepackage{lipsum} % Used for inserting dummy 'Lorem ipsum' text into the template

\usepackage[scale=0.75]{geometry} % Reduce document margins
%\setlength{\hintscolumnwidth}{3cm} % Uncomment to change the width of the dates column
%\setlength{\makecvtitlenamewidth}{10cm} % For the 'classic' style, uncomment to adjust the width of the space allocated to your name

%----------------------------------------------------------------------------------------
%	NAME AND CONTACT INFORMATION SECTION
%----------------------------------------------------------------------------------------

\firstname{杨} % Your first name
\familyname{志} % Your last name
% All information in this block is optional, comment out any lines you don't need
\title{认知神经科学博士,研究员}
\address{website: phi-group.top}{}
\mobile{(+86)18611710840}
\email{yangz@smhc.org.cn}
%\photo[70pt][0.4pt]{pictures/picture} % The first bracket is the picture height, the second is the thickness of the frame around the picture (0pt for no frame)
\quote{}

%----------------------------------------------------------------------------------------

\begin{document}
\makecvtitle % Print the CV title

\section{研究方向}
\cvitem{}{\textbf{神经影像方法学}}
\cvitem{}{\textbf{脑发展和儿童青少年精神病学}}

%----------------------------------------------------------------------------------------
%	EDUCATION SECTION
%----------------------------------------------------------------------------------------

\section{教育背景}
\cventry{2005--2007}{访问学生,美国Emory大学,生物医学工程专业}{导师:Xiaoping P. Hu教授}{}{}{}
\cventry{2003--2008}{硕博连读研究生, 中国科学院心理研究所,认知神经科学专业}{导师:翁旭初研究员}{}{}{}  % Arguments not required can be left empty
\cventry{1999--2003}{本科,清华大学,生物医学工程专业}{}{}{}{}

%------------------------------------------------

%\subsection{PROJECT}

%\cventry{2015}{Foot Step Power Generation by Piezoelectric Material}{Live Project.}{}{}{This is basically a Live Project. The main theme of the project is to generate power by applying pressure on piezoelectric material with the help of non-renewable resources like foot step and other.}

%----------------------------------------------------------------------------------------
%	Professional History
%----------------------------------------------------------------------------------------

\section{工作经历}
\cventry{2018--至今}{特聘研究员,博士生导师,教授委员会成员}{上海交通大学心理与行为科学研究院}{}{}{}
\cventry{2018--至今}{执行主任}{上海市精神卫生中心神经影像平台}{}{}{}
\cventry{2017--至今}{研究员,博士生导师,心理健康与神经影像研究组主任}{上海市精神卫生中心,上海交通大学医学院}{}{}{}
\cventry{2015-2018}{岗位教授}{中国科学院大学}{}{}{}
\cventry{2012-2017}{副研究员,博士生导师}{中国科学院心理研究所}{}{}{}
\cventry{2012-2014}{博士后}{美国国家精神卫生研究所,合作导师:Peter A. Bandettini研究员}{}{}{}
\cventry{2008-2012}{助理研究员}{中国科学院心理研究所}{}{}{}

%----------------------------------------------------------------------------------------
%	Funds
%----------------------------------------------------------------------------------------

\section{研究项目}
\cventry{2020-2023}{国家自然科学基金面上项目}{基于跨疾病神经影像数据库的精神障碍个体化辅助诊断研究(81971682)}{负责人,66万元}{}{}

\cventry{2016-2019}{国家自然科学基金面上项目 }{利用自然刺激下的神经影像探测精神分裂症的脑功能损伤亚型(81571756)}{负责人,67.2万元}{}{}

\cventry{2013-2016}{国家自然科学基金面上项目}{精神障碍的神经影像标志的特异性研究(81270023)}{负责人,70万元}{}{}

\cventry{2010-2012}{国家自然科学基金青年项目}{数据驱动的精神障碍分类神经影像标志研究(30900366)}{负责人,21万元}{}{}

\cventry{2020-2023}{上海市自然科学基金原创探索项目}{青少年味知觉与焦虑水平的关系和脑机制研究(20ZR1472800)}{负责人,50万元}{}{}

\cventry{2015-2017}{北京市科技新星科研项目}{精神疾病的神经影像标志:大数据时代的方法学(2015079B)}{负责人,35万元}{}{}

\cventry{2017-2020}{上海市教委高峰高原项目}{精神分裂症的神经影像标志研究(20171929)}{负责人,100万元}{}{}

\cventry{2018-2021}{上海市卫健委优秀学科带头人培养计划}{儿童青少年社交焦虑障碍的脑发育异常(2018BR17)}{负责人,90万元}{}{}

\cventry{2015-2019}{中国科学院青年创新促进会研究项目}{}{负责人,40万元}{}{}

\cventry{2014-2019}{中国科学院心理研究所杰出青年研究者项目}{基于脑网络的群体检测方法学研究}{负责人, 40万元}{}{}


%----------------------------------------------------------------------------------------
%	Publication
%----------------------------------------------------------------------------------------
\section{发表论文}
\cvitem{}{Jiang L, Qiao K, Li Q, Hu Y, Zhang X, Wang J, Peng D, Fan Q, Zhao M, Sheng J, Wang J, Li C, Fang Y, Wang Z*, \textbf{\underline{Yang Z*}} (2022): Categorical and dimensional deficits in hippocampal subfields among schizophrenia, obsessive-compulsive disorder, bipolar disorder, and major depressive disorder. \emph{Biological Psychiatry: Cognitive Neuroscience and Neuroimaging.} In press}


\cvitem{}{Zhang H, Li J, Su X, Hu Y, Liu T, Ni S, Li H, Zuo X-N, Fu J*, Yuan T-F*, \textbf{\underline{Yang Z*}} (2022): Growth charts of brain morphometry for preschool children. \emph{NeuroImage} 255: 119178.DOI: 10.1016/j.neuroimage.2022.119178}

\cvitem{}{Zhang Q, Li B, Jin S, Liu W, Liu J, Xie S, Zhang L, Kang Y, Ding Y, Zhang X, Cheng W*, \textbf{\underline{Yang Z*}} (2022): Comparing the Effectiveness of Brain Structural Imaging, Resting-state fMRI, and Naturalistic fMRI in Recognizing Social Anxiety Disorder in Children and Adolescents. \emph{Psychiatry Research: Neuroimaging} 111485. DOI: 10.1016/j.pscychresns.2022.111485}

\cvitem{}{Ding Y, Liu J, Zhang X, \textbf{\underline{Yang Z*}} (2022): Dynamic Tracking of State Anxiety via Multi-Modal Data and Machine Learning. \emph{Frontiers in Psychiatry} 13: 757961. DOI: 10.3389/fpsyt.2022.757961}

\cvitem{}{Hu Y, \textbf{\underline{Yang Z*}} (2021). Impact of inter-individual variability on the estimation of default mode network in temporal concatenation group ICA. \emph{Neuroimage}. 234: 118114. DOI: 10.1016/j.neuroimage.2021.118114}

\cvitem{}{Xie S, Zhang X*, Cheng W, \textbf{\underline{Yang Z}} (2021): Adolescent anxiety disorders and the developing brain: comparing neuroimaging findings in adolescents and adults. \emph{General Psychiatry} 34: e100411. DOI: 10.1136/gpsych-2020-100411}

\cvitem{}{Li Q, Jiang L, Qiao K, Hu Y, Chen B, Zhang X, Ding Y, \textbf{\underline{Yang Z*}}, Li C. (2021): INCloud: integrated neuroimaging cloud for data collection, management, analysis and clinical translations. \emph{General Psychiatry} 34: e100651. DOI: 10.1136/gpsych-2021-100651}

\cvitem{}{Jiang L, Cui H, Zhang C, Cao X, Gu N, Zhu Y, Wang J, \textbf{\underline{Yang Z}}, Li C. (2021): Repetitive Transcranial Magnetic Stimulation for Improving Cognitive Function in Patients With Mild Cognitive Impairment: A Systematic Review. \emph{Frontiers in Aging Neuroscience} 12: 593000. DOI: 10.3389/fnagi.2020.593000. eCollection 2020}

\cvitem{}{Zeng Y, Tao F, Cui Z, Wu L, Xu J, Dong W, Liu C, \textbf{\underline{Yang Z}}, Qin S*. (2021): Dynamic integration and segregation of amygdala subregional functional circuits linking to physiological arousal. \emph{NeuroImage} 238: 118224. DOI: 10.1016/j.neuroimage.2021.118224}

\cvitem{}{Xu Z, Zhang X, Chang H, Kong Y, Ni Y, Liu R, Zhang X, Hu Y, \textbf{\underline{Yang Z}}, Hou M, Mao R, Liu W-T, Du Y, Yu S, Wang Z, Ji M*, Zhou Z*. (2021): Rescue of maternal immune activation-induced behavioral abnormalities in adult mouse offspring by pathogen-activated maternal Treg cells. \emph{Nature Neuroscience} 24(6): 818–830. DOI: 10.1038/s41593-021-00837-1}

\cvitem{}{Gao J, Chen G, Wu J, Wang Y, Hu Y, Xu T, Zuo X-N*, \textbf{\underline{Yang Z*}} (2020). Reliability map of individual differences reflected in inter-subject correlation in naturalistic imaging. \emph{Neuroimage} 223: 117277. DOI: 10.1016/j.neuroimage.2020.117277}

\cvitem{}{Liu Z, Hu Y, Zhang Y, Liu W, Zhang L, Wang Y, Yang H, Wu J, Cheng W*, \textbf{\underline{Yang Z*}} (2020). Altered gray matter volume and structural co-variance in adolescents with social anxiety disorder: evidence for a delayed and unsynchronized development of the fronto-limbic system. \emph{Psychological Medicine} 51(10): 1742-1751. DOI: 10.1017/S0033291720000495}

\cvitem{}{Guo Q, Hu Y, Zeng B, Tang Y, Li G, Zhang T, Wang J, Northoff G, Li C, Goff D, Wang J*, \textbf{\underline{Yang Z*}} (2020). Parietal memory network and default mode network in first-episode drug-na{\"\i}ve schizophrenia: Associations with auditory hallucination.\emph{Human Brain Mapping} 41(8): 1973-1984. DOI: 10.1002/hbm.24923}

\cvitem{}{Zhang Y, Liu W, Lebowitz ER, Zhang F, Hu Y, Liu Z, Yang H, Wu J, Wang Y, Silverman WK, \textbf{\underline{Yang Z*}}, Cheng W* (2020). Abnormal asymmetry of thalamic volume moderates stress from parents and anxiety symptoms in children and adolescents with social anxiety disorder. \emph{Neuropharmacology} 180: 108301. DOI: 10.1016/j.neuropharm.2020.108301}

\cvitem{}{\textbf{\underline{Yang Z*}}, Wu J, Xu L, Deng Z, Tang Y, Gao J, Hu Y, Zhang Y, Qin S*, Li C, Wang J* (2020). Individualized psychiatric imaging based on inter-subject neural synchronization in movie watching. \emph{Neuroimage}: 116227. DOI:10.1016/j.neuroimage.2019.116227}

\cvitem{}{Cui H, Zhang B, Li W, Li H, Pang J, Hu Q, Zhang L, Tang Y, \textbf{\underline{Yang Z}}, Wang J, Li C*. (2020): Insula shows abnormal task-evoked and resting-state activity in first-episode drug-naïve generalized anxiety disorder. \emph{Depression and Anxiety} 37: 632–644. DOI: 10.1002/da.23009}

\cvitem{}{Deng Z, Wu J, Gao J, Hu Y, Zhang Y, Wang Y, Dong H, \textbf{\underline{Yang Z*}}, Zuo, X-N (2019). Segregated precuneus network and default mode network in naturalistic imaging. \emph{Brain Structure and Function} 224(9): 3133-3144. DOI: 10.1007/s00429-019-01953-2.}

\cvitem{}{Zhang Y, Xu Li, Hu Y, Wu J, Li C, Wang J*, \textbf{\underline{Yang Z*}} (2019). Functional connectivity between sensory-motor sub-networks reflects the duration of untreated psychosis and predicts treatment outcome of first-episode drug-na{\"\i}ve schizophrenia. \emph{Biological Psychiatry: Cognitive Neuroscience and Neuroimaging}. 4(8): 697-705}

\cvitem{}{Jiang L, Cao X, Jiang J, Li T, Wang J, {\underline{\textbf{Yang Z*}}}, Li C* (2019). Atrophy of hippocampal subfield CA2/3 in healthy elderly men is related to educational attainment. \emph{Neurobiology of Aging} 80: 21-28}

\cvitem{}{Hu Y, Du W, Zhang Y, Li N, Han Y*, \textbf{\underline{Yang Z*}} (2019). Loss of parietal memory network integrity in Alzheimer's disease. \emph{Frontiers in Aging Neuroscience} 11: 67}

\cvitem{}{Zhao Q, Xu T, Wang Y, Chen D, Liu Q, \textbf{\underline{Yang Z*}}, Wang Z*. (2019). Limbic cortico-striato-thalamo-cortical functional connectivity in drug-na{\"\i}ve patients of obsessive-compulsive disorder. \emph{Psychological Medicine} 49: 1156-1165. DOI: 10.1017/S0033291719002988}

\cvitem{}{Wang J, Hu Y, Li H, Ge L, Li J, Cheng L, \textbf{\underline{Yang Z*}}, Zuo XN, Xu Y* (2018). Connecting openness and the resting-state brain network: A discover-validate approach. \emph{Frontiers in Neuroscience} 12: 762.}

\cvitem{}{\textbf{\underline{Yang Z}}, Zuo XN*, McMahon KL, Craddock RC, Kelly C, de Zubicaray GI, Hickie I, Bandettini PA, Castellanos FX, Milham MP*, Wright MJ (2016). Genetic and environmental contributions to functional connectivity architecture of the human brain. \emph{Cerebral Cortex} 26: 2341-2352}

\cvitem{}{\textbf{\underline{Yang Z*}}, Qiu J, Wang P, Liu R, Zuo X* (2016). Brain structure-function associations identified in large-scale neuroimaging data. \emph{Brain Structure \& Function} 221: 4459-4474}

\cvitem{}{Hu Y, Wang J, Li C, Wang Y-S, \textbf{\underline{Yang Z*}}, Zuo X-N (2016). Segregation between the parietal memory network and the default mode network: effects of spatial smoothing and model order in ICA. \emph{Science Bulletin}. 61 (24): 1844-1854}

\cvitem{}{\textbf{\underline{Yang Z*}}, Huang Z, Gonzalez-Castillo J, Dai R, Northoff G, Bandettini P (2014). Using fMRI to decode true thoughts independent of intention to conceal. \emph{NeuroImage} 99: 80-92}

\cvitem{}{\textbf{\underline{Yang Z*}}, Chang C, Xu T, Jiang L, Handwerker D, Castellanos F, Milham M, Bandettini P, Zuo X* (2014). Connectivity trajectory across lifespan differentiates the precuneus from the default network. \emph{NeuroImage} 89: 45-56}

\cvitem{}{\textbf{\underline{Yang Z*}}, Zuo X, Wang P, Li Z, Laconte S, Bandettini PA, Hu X (2012). Generalized RAICAR: Discover homogeneous subject (sub)groups by reproducibility of their intrinsic connectivity networks. \emph{NeuroImage} 63: 403-414}

\cvitem{}{\textbf{\underline{Yang Z}}, Xu Y*, Xu T, Hoy C, Handwerker D, Chen G, Northoff G, Zuo X*, Bandettini P (2014). Brain network informed subject community detection in early-onset schizophrenia. \emph{Scientific Reports} 4: 5549}

\cvitem{}{\textbf{\underline{Yang Z}}, LaConte S, Weng X, Hu X* (2008). Ranking and averaging independent component analysis by reproducibility (RAICAR).  \emph{Human Brain Mapping} 29: 711-725}

\cvitem{}{Xu T, \textbf{\underline{Yang Z (co-first author)}}, Jiang L, Xing XX, Zuo XN* (2015). A connectome computation system for discovery science of brain. \emph{Science Bulletin} 60: 86-95}

\cvitem{}{\textbf{\underline{Yang Z*}}, Fang F, Weng X (2012). Recent developments in multivariate pattern analysis for functional MRI.  \emph{Neuroscience Bulletin} 28: 399-408}

\cvitem{}{\textbf{\underline{Yang Z*}}, Wu P, Weng X, Bandettini P (2014). Cerebellum engages in automation of verb-generation skill.  \emph{Journal of Integrative Neuroscience} 13: 1-17}

\cvitem{}{Liu C, Li F, Li S, Wang Y, Tie C, Wu H, Zhou Z, Zhang D, Dong J, \textbf{\underline{Yang Z*}}, Wang C* (2012). Abnormal baseline brain activity in bipolar depression: A resting-state functional magnetic resonance imaging study. \emph{Psychiatry Research: Neuroimaging} 203: 175-179}

\cvitem{}{Tang L, Liu C, Jing B, Ma X, Li H, Zhang Y, Li F, Wang Y, \textbf{\underline{Yang Z*}}, Wang C* (2014). Voxel-based morphometry study of the insular cortex in bipolar depression. \emph{Psychiatry Research: Neuroimaging} 224: 89-95}

\cvitem{}{Liu C, Ma X*, Wu X, Zhang Y, Zhou F, Li F, Tie C, Dong J, Wang Y, \textbf{\underline{Yang Z*}}, Wang C (2013). Regional homogeneity of resting-state brain abnormalities in bipolar and unipolar depression. \emph{Progress in Neuro-Psychopharmacology \& Biological Psychiatry} 41: 52-59}

\cvitem{}{\textbf{\underline{Yang Z}}, Zhao J, Jiang Y*, Li C, Wang J, Weng X*, Northoff G (2011). Altered negative Unconscious processing in major depressive disorder: An exploratory neuropsychological study.  \emph{PLoS One} 6: e21881}

\cvitem{}{Li W, Cui H, Zhu Z, Kong L, Guo Q, Zhu Y, Hu Q, Zhang L, Li H, Li Q, Jiang J, Meyers J, Li J, Wang J*, \textbf{\underline{Yang Z*}}, Li C* (2016). Aberrant functional connectivity between the amygdala and the temporal pole in drug-free generalized anxiety disorder. \emph{Frontiers in Human Neuroscience} 10: 549}

\cvitem{}{Sun L, Xu H, Zhang J, Li W, Nie J, Qiu Q, Liu Y, Fang Y, \textbf{\underline{Yang Z*}}, Li X* and Xiao S* (2018). Alcohol consumption and subclinical findings on cognitive function, biochemical indexes, and cortical anatomy in cognitively normal aging Han Chinese population. \emph{Frontiers in Aging Neuroscience} 10: 182}

\cvitem{}{Xu G, Jiang Y, Ma L, \textbf{\underline{Yang Z*}}, Weng X* (2012). Similar spatial patterns of neural coding of category selectivity in FFA and VWFA under different attention conditions. \emph{Neuropsychologia} 50: 862-868}

\cvitem{}{Huang Z, Zhang X, \textbf{\underline{Yang Z*}}, Dong G, Wu J, Chan A, Weng X (2010). Verbal memory retrieval engages visual cortex in musicians. \emph{Neuroscience} 168: 179-189}

%----------------
% patents
%----------------
\section{发明专利}

\cventry{2022}{针对精神疾病实现神经影像辅助诊断处理的系统}{申请号:202111212750.1}{第一发明人, 已授权}{}{}{}

\cventry{2022}{基于健康人群分布实现脑影像特征归一化处理的方法、系统、装置、处理器及其存储介质}{申请号:202111208007.9}{第一发明人, 已授权}{}{}{}

\cventry{2022}{基于单中心标定数据实现医学图像域自适应分割的方法、系统、装置、处理器及其存储介质}{申请号:202110757216.2}{第一发明人, 已授权}{}{}{}

\cventry{2018}{一种实时测量和反馈人际沟通效率的装置}{专利号:ZL20161 0114680.9}{第一发明人, 已授权,已转化为产品}{}{}{}




%----------------------------------------------------------------------------------------
%	Other publications
%----------------------------------------------------------------------------------------
\section{其他著作}

\cventry{2017}{脑科学与课堂:以脑为导向的教学模式}{玛丽亚·哈迪曼 著,\textbf{\underline{杨志}}、王培培等 译,华东师范大学出版社}{}{}{}
\cventry{2019}{给孩子的脑科学实验室}{埃里克·H·查德勒   著,\textbf{\underline{杨志}} 译,华东师范大学出版社}{}{}{}


%----------------
% project experience
%----------------
\section{研究平台建设经验}
\cventry{2017-至今}{INCloud神经影像云}{实现神经影像数据的获取、存储、管理、分析、分享的全流程云系统,建成精神疾病脑影像自动化处理流水线PhiPipe、WeProMRI、脑特征数据库和知识图谱}{创始者、负责人}{}{}

\cventry{2020-至今}{精神疾病的神经影像辅助诊断系统}{连接神经影像研究技术与临床实践,实现自动化的结构化脑形态异常报告及多类型精神疾病的人工智能辅助诊断系统,目前已投入临床应用测试}{创始者、负责人}{}{}

\cventry{2020-至今}{科研神经影像中心}{建成面向脑科学研究需要的多模态神经影像设施,实现磁共振影像、眼动、脑电、生理监测数据的同步采集,全面管理和维护设施的安全高效运行}{创始者、负责人}{}{}



%----------------------------------------------------------------------------------------
%	Teaching
%----------------------------------------------------------------------------------------
\section{教学经验}
\cventry{2022-2023}{脑功能成像:原理与实践(研究生)}{}{32学时/学年}{上海交通大学}{}
\cventry{2018-2022}{普通心理学(本科生)}{}{12学时/学年}{上海交通大学/上海交通大学医学院}{}
\cventry{2018-2022}{心理学研究方法(研究生)}{}{6学时/学年}{上海交通大学/上海交通大学医学院}{}
\cventry{2018-2022}{心理学与神经科学(研究生)}{}{3学时/学年}{上海交通大学/上海交通大学医学院}{}
\cventry{2019-2022}{科研伦理和写作(研究生)}{}{6学时/学年}{上海交通大学/上海交通大学医学院}{}
\cventry{2015-2016}{认知心理学(研究生)}{}{6学时/学年}{中国科学院大学}{}
\cventry{2012-2018}{高级统计和机器学习(研究生)}{}{60学时/学期}{中国科学院心理研究所}{}
\cventry{2008-2018}{心理统计学(研究生)}{}{60学时/学期}{中国科学院心理研究所}{}
\cventry{2008-2018}{脑发育和学习(研究生)}{}{16学时/学期}{中国科学院心理研究所}{}
\cventry{2008-2010}{认知神经科学(研究生)}{}{16学时/学期}{中国科学院心理研究所}{}

%----------------------------------------------------------------------------------------
%	affiliations
%----------------------------------------------------------------------------------------
\section{学术任职}
\cventry{2021--至今}{兼职研究员}{上海科技大学}{}{}{}
\cventry{2020--至今}{兼职教授}{北京邮电大学}{}{}{}
\cventry{2020--至今}{校外研究生导师}{首都师范大学}{}{}{}
\cventry{2020--至今}{校外研究生导师}{复旦大学}{}{}{}
\cventry{2019--至今}{特聘研究员}{上海陈天桥国际脑疾病研究所}{}{}{}
\cventry{2019--至今}{委员}{中国康复医学会脑功能检测与调控专业委员会脑功能检测与康复学组}{}{}{}
\cventry{2018--至今}{副主任委员}{吴阶平医学基金会认知障碍多学科诊疗专家委员会}{}{}{}
\cventry{2018--至今}{委员}{中国心理学会体育运动心理专业委员会运动认知神经科学学组}{}{}{}
\cventry{2020--至今}{Associate Editor}{Brain Informatics}{}{}{}
\cventry{2018--至今}{Associate Editor}{Frontiers in Neuroscience: Brain Imaging Methods}{}{}{}
\cventry{2018--至今}{编委}{心理学通讯}{}{}{}
%----------------------------------------------------------------------------------------
%   Examples
%----------------------------------------------------------------------------------------
%\renewcommand{\listitemsymbol}{-~}
%\cvlistitem{Quick Learner.}
%\cvlistitem{Able to handle multiple situations at the same time.}
%\cvlistitem{Easily Adaptable.}
%\cvlistitem{Can manage time effectively}
%\cvitemwithcomment{Hindi}{Intermediate}{Can understand good}
%\cvitemwithcomment{English}{Fluent}{Conversationally fluent}

%----------------------------------------------------------------------------------------
%	Award
%----------------------------------------------------------------------------------------

\section{获奖和荣誉}

\cvitem{2018}{\textbf{入选上海市卫健委“优秀学科带头人培养计划”}}
\cvitem{2017}{\textbf{获北京市科技进步二等奖(第二完成人)}}
\cvitem{2017}{\textbf{入选上海交通大学“晨光优秀学者奖励计划”}}
\cvitem{2017}{\textbf{入选上海交通大学医学院“双百人计划”}}
\cvitem{2016}{\textbf{入选中国科学院青年创新促进会}}
\cvitem{2015}{\textbf{获教育部科技进步一等奖(第八完成人)}}
\cvitem{2015}{\textbf{入选北京市“科技新星计划”}}
\cvitem{2014}{\textbf{获中国科学院心理研究所“杰出青年研究者”奖}}
\cvitem{2009--2011}{\textbf{中国科学院心理研究所“优秀助理研究员”}}




%----------------------------------------------------------------------------------------
%	INTERESTS SECTION
%----------------------------------------------------------------------------------------

%\section{Interests}

%\renewcommand{\listitemsymbol}{-~} % Changes the symbol used for lists

%\cvlistdoubleitem{Driving}{Photography}
%\cvlistdoubleitem{Cooking}{Basket Ball}
%\cvlistitem{Cricket}
%\section{Personal Details}
%\cvitem{DOB}{04th of Feb, 1994}
%\cvitem{Address}{D.No:59-20-7/4B,APT Backside,kakinada-533002}
%\cvitem{Languages}{Telugu,Hindi,English}
%\cvitem{Mobile No.}{+91 9492390759}
%\section{References}
%\cvitem{Name}{K.Harish}
%\cvitem{Designation}{Senior Software Engineer}
%\cvitem{Organisation}{Happiest Minds-Bangalore}
%\cvitem{Ph No}{8105543434}
%\cvitem{Mai}{harish.kmail@gmail.com}


%----------------------------------------------------------------------------------------
%	COVER LETTER
%----------------------------------------------------------------------------------------

% To remove the cover letter, comment out this entire block

%\clearpage

%\recipient{HR Department}{Corporation\\123 Pleasant Lane\\12345 City, State} % Letter recipient
%\date{\today} % Letter date
%\opening{Dear Sir or Madam,} % Opening greeting
%\closing{Sincerely yours,} % Closing phrase
%\enclosure[Attached]{curriculum vit\ae{}} % List of enclosed documents

%\makelettertitle % Print letter title

%\lipsum[1-3] % Dummy text

%\makeletterclosing % Print letter signature

%----------------------------------------------------------------------------------------

\end{document}